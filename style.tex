\usepackage{amsmath} 
\usepackage{amssymb}
\usepackage{amsfonts}
\usepackage{mathtools}
\usepackage{amsthm}
\usepackage{enumitem}
\usepackage[a4paper, left=3cm, right=3cm, top=3cm, bottom = 2cm]{geometry} % borders
\usepackage{dsfont}
\usepackage{extarrows}
\usepackage[mathscr]{euscript}
\usepackage{wasysym}
\usepackage{fancyhdr} % for customizing header / footer
\usepackage{tcolorbox}
\usepackage{imakeidx} % for index pages
\tcbuselibrary{theorems}
\usepackage{faktor} % for quotient spaces

\usepackage{graphicx}
%\usepackage[dvipsnames]{xcolor}

\usepackage{tikz}
\usepackage{wrapfig}

\usepackage{algorithm,algorithmic} % for pseudo-codes, like gram schmidt

\usepackage{tkz-euclide} %for tikz picture
\tikzset{point/.style={insert path={ node[scale=4*sqrt(\pgflinewidth)]{.} }}} % for tikz points
\newcommand{\tdot}[4]{\draw[very thick,text=black] (#1,#2) -- node[point,#4]{#3}(#1,#2);} % draw point at specific location with a tag
\newcommand{\reddot}[4]{\draw[very thick,text=black,red] (#1,#2) -- node[point,#4]{#3}(#1,#2);}
\newcommand{\bluedot}[4]{\draw[very thick,text=black,blue] (#1,#2) -- node[point,#4]{#3}(#1,#2);}

\newcommand{\myeq}[2]{\stackrel{\mathclap{\normalfont\mbox{#1}}}{#2}} % for putting things like a reference over a math operator like "="
\newcommand{\limn}{\displaystyle{\lim_{n \to \infty}}} % limes, as n tends to infinity
\newcommand{\logg}{\text{log }} % writing log as not recursive, but with a space afterwards
\newcommand{\rec}[1]{\emph{#1}} % write recursive
\newcommand{\bol}[1]{\textbf{#1}} % write bold

%% ARROWS
\newcommand{\rarr}{\rightarrow}
\newcommand{\Rarr}{\Rightarrow}
\newcommand{\Lrarr}{\Leftrightarrow}
\newcommand{\lorarr}{\longrightarrow}
\newcommand{\Lorarr}{\Longrightarrow}
\newcommand{\Lolrarr}{\Longleftrightarrow}
\newcommand{\weak}{\rightharpoonup}
\newcommand{\weakstar}{\myeq{\scriptsize$\ast$}{\weak}}
\newcommand{\wweak}{\myeq{\scriptsize w}{\lorarr}}
\newcommand{\strong}{\myeq{\scriptsize s}{\lorarr}}

%% MATHEMATICAL LETTERS
\newcommand{\N}{\mathbb{N}} % natural numbers
\newcommand{\NN}{\mathscr{N}}
%\newcommand{\Compl}{\mathbb{C}} % complex numbers
\newcommand{\R}{\mathbb{R}} % real numbers
\newcommand{\RR}{\overline{\R}}
\newcommand{\Z}{\mathbb{Z}} % whole numbers
\newcommand{\Q}{\mathbb{Q}} % rational numbers
%\newcommand{\U}{U\subset\R^d}
%\newcommand{\oneF}{\mathbbm{1}}
%\newcommand{\e}{\text{exp}}
%\renewcommand{\d}{d}
%\newcommand{\sin}{\text{sin}}
%\newcommand{\cos}{def}
\newcommand{\A}{\mathscr{A}}
%\newcommand{\E}{\mathscr{E}}
\newcommand{\F}{\mathscr{F}}
\newcommand{\B}{\mathscr{B}}
\let\oldH\H
\renewcommand{\H}{\mathscr{H}}
\newcommand{\C}{\mathbb{C}}
\newcommand{\K}{\mathbb{K}}
\newcommand{\LL}{\mathcal{L}}
\newcommand{\LLL}{\mathscr{L}}
%\newcommand{\M}{\mathcal{M}}
%\renewcommand{\O}{\mathcal{O}}
\renewcommand{\P}{\mathscr{P}}
\newcommand{\J}{\mathscr{J}}
%\newcommand{\W}{\mathscr{W}}
\newcommand{\D}{\mathcal{D}}
%\newcommand{\Ri}{\mathscr{R}}
\newcommand{\coll}{\mathscr{C}}
\newcommand{\G}{\mathcal{G}}
\newcommand{\OO}{\mathcal{O}}

%% SUMS, PRODUCTS, UNIONS, INTERSECTIONS
\newcommand{\summ}{\sum\limits}
\newcommand{\cupp}{\bigcup\limits}
\newcommand{\cuppn}{\bigcup_{n\in\N}}
\newcommand{\cuppj}{\bigcup\limits_{j=1}^\infty}
\newcommand{\cuppk}{\bigcup\limits_{k=1}^\infty}
\newcommand{\cuppjn}{\bigcup\limits_{j=1}^n}
\newcommand{\cuppkn}{\bigcup\limits_{k=1}^n}
\newcommand{\capp}{\bigcap\limits}
\newcommand{\cappjn}{\bigcap\limits_{j=1}^n}
\newcommand{\cappkn}{\bigcap\limits_{k=1}^n}
\newcommand{\disjoint}{\spa\cdot\hspace{-10.5pt}\bigcup\limits}
\DeclareMathOperator{\essup}{essup}
\DeclareMathOperator{\sgn}{sgn}
\DeclareMathOperator{\Ker}{Ker}
\DeclareMathOperator{\Ran}{Ran}

%% FUNCTIONS
\newcommand{\func}[3]{#1\colon#2 \lorarr #3}
%\newcommand{\grad}{\text{grad}}

%% LIMITS
\newcommand{\limes}[1]{\lim\limits_{#1\rightarrow\infty}}
\newcommand{\limesx}[2]{\lim\limits_{#1\rightarrow{}#2}}
\newcommand{\limessup}[1]{\limsup\limits_{#1\rightarrow\infty}}
\newcommand{\limessupx}[2]{\limsup\limits_{#1\rightarrow{}#2}}
\newcommand{\limesinfx}[2]{\liminf\limits_{#1\rightarrow{}#2}}
\newcommand{\limesinf}[1]{\liminf\limits_{#1\rightarrow\infty}}
\newcommand{\liminfn}{\liminf\limits_{n\to\infty}}
\newcommand{\limsupn}{\limsup\limits_{n\to\infty}}

%\newcommand{\AnN}{(A_n)_{n\in\N}}
%\newcommand{\An}{(A_n)_{n}}
%\newcommand{\BnN}{(B_n)_{n\in\N}}
%\newcommand{\Bn}{(B_n)_{n}}
%\newcommand{\AjN}{(A_j)_{j\in\N}}
%\newcommand{\Aj}{(A_j)_{j}}
%\newcommand{\BjN}{(B_j)_{j\in\N}}
%\newcommand{\Bj}{(B_j)_{j}}
\newcommand{\si}{$\sigma$-algebra }

%\renewcommand{\S}{\mathcal{S}}
\newcommand{\bbox}[1]{\fbox{$\displaystyle  #1 $}}
\newcommand{\series}[2]{\sum\limits_{#1=#2}^\infty}
\newcommand{\seriesn}[1]{\sum\limits_{n=#1}^\infty}
\newcommand{\cupn}{\bigcup\limits_{n=1}^\infty}
\newcommand{\intt}{\int\limits}
\newcommand{\fn}{(f_n)_n}
\newcommand{\GL}{\text{GL}}
\newcommand{\Cn}{\mathbb{C}^{n,n}}

\newcommand{\fuc}{f:U\longrightarrow\mathbb{C}}
\newcommand{\fxc}{ \func{f}{X}{ \mathbb{C} } }
\newcommand{\fxr}{\func{f}{X}{\overline{\R}}}
\DeclareMathOperator{\IM}{Im}
\DeclareMathOperator{\RE}{Re}
\newcommand{\supp}{\text{supp}}
\newcommand{\codim}{\text{codim}}
\newcommand{\summjn}{\sum\limits_{j=1}^n}
\newcommand{\supn}{\sup\limits_{n\in\N}}

\newcommand{\fC}{f:\mathbb{C}\longrightarrow\mathbb{C}}
\newcommand{\z}{\overline{z}}
\newcommand{\x}{\widetilde{x}}
\newcommand{\y}{\widetilde{y}}
\newcommand{\diam}{\text{diam}}
\newcommand{\gabu}{\func{\gamma}{[a,b]}{U}}
\newcommand{\gabc}{\func{\gamma}{[a,b]}{\C}}
\newcommand{\oa}{\overline{\alpha}}
\newcommand{\wid}{\text{\lightning}}
\newcommand{\Bew}{\bol{\rec{Beweis:}} }
\newcommand{\Res}{\text{Res}}
\newcommand{\limm}{\lim\limits}
\newcommand{\ddt}{\frac{d}{dt}}
\newcommand{\ddz}{\frac{d}{dz}}
\newcommand{\dydt}{\frac{dy}{dt}}
\newcommand{\dds}{\frac{d}{ds}}
\newcommand{\dPdt}{\frac{dP}{dt}}
\newcommand{\dpdt}{\frac{dp}{dt}}

\newcommand{\te}[1]{\text{#1}}
\newcommand{\loc}{\text{loc}}
\newcommand{\xn}{(x_n)_n}
\newcommand{\An}{(A_n)_n}
\newcommand{\Bn}{(B_n)_n}
\newcommand{\elln}{(\ell_n)_n}
\renewcommand{\S}{\mathcal{S}}
\newcommand{\E}{\mathcal{E}}

\renewcommand{\l}{\ell}
%\newcommand{\GL}{\text{GL}}
\newcommand{\tl}[1]{\tag{#1}\label{#1}}
\newcommand{\f}{\widehat{f}}

\newcommand{\fmn}{\func{f}{M}{N}}
\newcommand{\scal}[2]{\langle #1, #2\rangle}
\newcommand{\scalvvr}{\func{\scal{\cdot}{\cdot}}{V\times V}{\R}}
\newcommand{\scalvvc}{\func{\scal{\cdot}{\cdot}}{V\times V}{\C}}
\newcommand{\conj}[1]{\overline{#1}}
\newcommand{\conv}{\text{conv}}
\DeclareMathOperator{\dist}{dist}
\newcommand{\PHA}{\func{P}{\H}{A}}
\newcommand{\TXY}{\func{T}{X}{Y}}
\newcommand{\JXX}{\func{J}{X}{X^{**}}}




\newtheoremstyle{note}% style name
{2ex}% above space
{2ex}% below space
{}% body font
{}% indent amount
{}%{\scshape}% head font
{}% post head punctuation
{\newline}% post head punctuation
{}% head spec 

\newtheoremstyle{tstyle}% name of the style to be used
%{\topsep}% measure of space to leave above the theorem. E.g.: 3pt
%{\topsep}% measure of space to leave below the theorem. E.g.: 3pt
%{\itshape}% name of font to use in the body of the theorem
%{0pt}% measure of space to indent
%{\bfseries}% name of head font
%{. ---}% punctuation between head and body
%{ }% space after theorem head; " " = normal interword space
%{\thmname{#1}\thmnumber{ #2}\thmnote{ (#3)}}
{2ex}% measure of space to leave above the theorem. E.g.: 3pt
{2ex}% measure of space to leave below the theorem. E.g.: 3pt
{}% name of font to use in the body of the theorem
{}% measure of space to indent
{\bfseries}% name of head font
{\newline}%
{ }% punctuation between head and body
{ }% space after theorem head; " " = normal interword space
%{\thmname{#1}\thmnumber{ #2}\thmnote{ (#3)}}




%\theoremstyle{definition} % stattdessen plain / remark
%\theoremstyle{tstyle}

\newtheorem{mydef}{Definition}[section]

\newtheorem{satz}[mydef]{Satz}
%\theoremstyle{marginbreak}
\newtheorem{lem}[mydef]{Lemma}
\newtheorem{cor}[mydef]{Corollary}
\newtheorem*{bem}{Bemerkung}
\newtheorem*{bsp}{Beispiel}
\newtheorem{thm}[mydef]{Theorem}
\newtheorem{definition}[mydef]{Definition}
\newtheorem*{definition*}{Definition}
\newtheorem{prop}[mydef]{Proposition}

\makeindex[columns=3, title=Alphabetical Index,intoc] % for index pages	

%% FOOTER
%\pagestyle{headings} % put page-numbering on top of page and add chapter
\pagestyle{fancy}
\fancyhf{}
\renewcommand{\headrulewidth}{0pt} % remove header line
%\fancyhead[RE,LO]{\thepage}
%\fancyhead[RO]{\rec{\leftmark}}
%\fancyhead[LE]{\rec{\MakeUppercase\rightmark}}
\fancyhead[RO,LE]{\thepage}
\fancyhead[LO]{\rec{\MakeUppercase\rightmark}}
\fancyhead[RE]{\rec{\leftmark}}

%% HYPERREF - always load as last (or one of last) package
\usepackage{hyperref}
%	coloring links blue instead of making a red box around them
%\hypersetup{
%	colorlinks = true,
%	linkcolor = blue
%}

\endinput