\section{$L^p$ spaces. The basics}

Some notation: In the following we will consider measure spaces, i.e., a triple $(\Omega,\A,\mu)$, where $\Omega$ is a set (often a nice set like $\R^d$ with the topology of open sets) and
\begin{enumerate}[label=\arabic*)]
	\item $\A$ is a \rec{\bol{\si}} in $\Omega$, that is, a collection of subsets of $\Omega$ (i.e., $\A\subset\P(\Omega)$) such that
	\begin{enumerate}[label=1.\alph*)]
		\item $\emptyset\in\A$,
		\item $A\in\A\Lorarr A^C=\Omega\setminus A\in\A$,
		\item $\An\subset\A$, then 
		\[\cupp_{n\in\N}A_n\in\A.\tag{10.1}\label{10.1}\]
	\end{enumerate}
	We call the sets in $\A$ \rec{\bol{measurable}} and 1.c) says that countable unions of measurable sets are measurable.
	
	\item $\mu$ is a \rec{\bol{measure}} (on $\A$, or on $\Omega$), that is $\func{\mu}{\A}{[0,\infty]}$ with
	\begin{enumerate}[label=2.\alph*)]
		\item $\mu(\emptyset)=0$
		\item If $\An\subset\A$ are pointwise disjoint, $A_n\cap A_m=\emptyset$ for all $n\neq m$, then
		\[\mu\Big(\cupp_{n\in\N}A_n\Big)=\sum_{n\in\N}\mu(A_n)\;\;\;(\sigma\te{-additivity}).\tag{10.2}\label{10.2}\]
	\end{enumerate} 
	We also (often) assume that the measure $\mu$ is \rec{\bol{$\sigma$-finite}}, that is,
	\[\exists E_n\in\A,n\in\N\te{ such that }\cuppn E_n=\Omega\te{ and }\mu(E_n)<\infty\;\forall n\in\N.\tag{10.3}\label{10.3}\]
	
	A set $N\subset\A$ is a \rec{\bol{null set}}, if $\mu(N)=0$ and a property \rec{\bol{holds a.e.}} (almost everywhere) (or $\mu$-almost everywhere, $\mu$-a.e.) if there exists a null set $N$ such that it holds everywhere on $N^C=\Omega\setminus N$.\vspace{1.5mm}
	
	A function $\func{f}{\Omega_1}{\Omega_2}$ is called \rec{\bol{measurable}} if $\A_2$ is a \si on $\Omega_2$ and $\A_1$ a \si on $\Omega_1$ and 
	\[f^{-1}(B)\in\A_1\;\;\;\forall B\in\A_2.\]
	We put $\LL^1(\mu)=\LL^1(\Omega,\mu)=\LL^1$ the space of integrable functions from $\Omega$ to $\R$ or from $\Omega$ to $\RR$, $\RR=\R\cup\{-\infty,\infty\}$ and we will often use the same symbol for integrable functions from $\Omega$ to $\C$. 
	
	Here $\R,\RR,\C$ have the usual Borel-\si, i.e., the \si generated by the open sets (see Analysis III). 
	
	The integral is denoted by
	\[\int fd\mu=\int f=\int_\Omega fd\mu,\]
	$\NN_1(f):=\|f\|_1:=\int|f|d\mu$.\vspace{2mm}
	
	\rec{Slight warning:} $\|\cdot\|_1$ is, strictly speaking, not a norm on $\LL^1(\mu)$, since $\|f\|_1=0$ does not imply that $f=0$, only that $f(x)=0$ $\mu$-a.e.
	
	We often "forget" this, by identifying functions which agree almost everywhere. Strictly speaking, we should consider the space
	\[L^1(\mu):=\faktor{\LL^1(\mu)}{\sim},\]
	a space of equivalence classes, for which one identifies functions, which are equal almost everyhwere:
	\[f\sim g\te{ if }\NN_1(f-g)=0\]
	(that is, $f\sim g$ iff there exists a null set $N\in\A$ s.t. $f(x)=g(x)\;\forall x\notin N$).
\end{enumerate}

\begin{thm}[Beppo-Levi, monotone convergence]\label{x.1}\index{measure theory!Beppo-Levi}
	Let $\fn$ be a sequence of non-negative functions which is increasing,
	\[0\leq f_1\leq f_2\leq\ldots\leq f_n\leq f_{n+1}\leq\ldots \te{ \rec{(}a.e.\rec{)}}\]
	Then \[f(x)=\limn f_n(x)=\supn f_n\]
	exists a.e. \rec{(}might be $+\infty$\rec{)} and
	\[\int fd\mu=\int\supn f_nd\mu=\supn\int f_nd\mu=\limn\int f_nd\mu.\tag{10.4}\label{10.4}\]
	Moreover, if $\fn$ is bounded in $\LL^1$, that is
	\[\supn\int f_nd\mu<\infty\tag{10.5}\label{10.5}\]
	then
	\[f\in\LL^1\te{ and }\limn\|f-f_n\|_1=0.\tag{10.6}\label{10.6}\]
\end{thm}

\begin{thm}[Dominated convergence, Lebesgue]\label{x.2}\index{measure theory!Lebesgue}
	Let $\fn\subset\LL^1(\mu)$ such that
	\begin{enumerate}[label=\alph*)]
		\item $f_n\lorarr f$ $(n\rarr\infty)$ $\mu$-a.e.
		\item $\exists g\in\LL^1,g\geq0$ such that $|f_n|\leq g$ $\mu$-a.e.
	\end{enumerate}
	Then 
	\[f\in\LL^1\te{ and }\limn\|f-f_n\|_1=0.\tag{10.7}\label{10.7}\]
\end{thm}

\begin{lem}[Fatou]\label{x.3}\index{measure theory!Fatou}
	Let $\fn$ be a sequence of functions $\func{f_n}{\Omega}{[0,\infty]}$. Then with $f=\liminfn f_n$, we have
	\[\int fd\mu=\int\liminfn f_nd\mu\leq\liminfn\int f_nd\mu.\tag{10.8}\label{10.8}\]
\end{lem}

\paragraph{Product measures}
	Let $(\Omega_1,\A_1,\mu_1),(\Omega_2,\A_2,\mu_2)$ be two $\sigma$-finite measure spaces
	\[\Omega:=\Omega_1\times\Omega_2,\;\;\;\A:=\A_1\otimes\A_2\]
	product $\sigma$-algebra, i.e., the \si generated by the sets
	\[\E:=\{A_1\times A_2\colon A_j\in\A_j,j=1,2\}.\]
	$\mu=\mu_1\otimes\mu_2$ defined on $\E$ by
	\[\mu(A_1\times A_2):=\mu_1(A_1)\mu_2(A_2)\]
	and extended to $\A_1\otimes\A_2=\sigma(\E)$ in the usual way (Carathéodory etc...)

\begin{thm}[Tonelli]\label{x.4}\index{measure theory!Tonelli}
	Let $\func{F}{\Omega_1\times\Omega_2}{[0,\infty]}$ be $\A_1\otimes\A_2$-measurable, $\mu_1,\mu_2$ $\sigma$-finite.
	Then
	\begin{enumerate}[label=\alph*)]
		\item \label{x.iv.a}\[\Omega_1\ni x\mapsto F(x,y)\te{ is }\A_1\te{-measurable for all }y\in\Omega_2.\]
		\[\Omega_2\ni y\mapsto F(x,y)\te{ is }\A_2\te{-measurable for all }x\in\Omega_1.\]
		\item \label{x.iv.b}\[\Omega_1\ni x\mapsto\int_{\Omega_2}F(x,y)\mu_2(dy)\te{ is }\A_1\te{-measurable.}\]
			\[\Omega_2\ni y\mapsto\int_{\Omega_1}F(x,y)\mu_1(dx)\te{ is }\A_2\te{-measurable.}\]
		\item \label{x.iv.c}	
			\[\int_{\Omega_1}\int_{\Omega_2}F(x,y)\mu_2(dy)\mu_1(dx)=\int_{\Omega_2}\int_{\Omega_1}F(x,y)\mu_1(dx)\mu_2(dy)=\int_{\Omega_1\times\Omega_2}Fd\mu_1\otimes\mu_2.\]
	\end{enumerate}
\end{thm}
\begin{thm}[Fubini]\label{x.5}\index{measure theory!Fubini}
	Let $(\Omega_1,\A_1,\mu_1),(\Omega_2,\A_2,\mu_2)$ be $\sigma$-finite measure spaces and $\func{u}{\Omega_1\times\Omega_2}{\RR}$ be $\A_1\otimes\A_2$-measurable. If $u$ is $\mu_1\otimes\mu_2$-integrable, i.e., $u\in\LL^1(\mu_1\otimes\mu_2)$ $\big(\int_{\Omega_1\times\Omega_2}|u|d\mu_1\otimes\mu_2<\infty\big)$. Then
	\begin{enumerate}[label=\alph*)]
		\item \label{x.5.a}
		\begin{equation*}
			\tag{10.9}\label{10.9}
			\begin{aligned}
				x\mapsto u(x,y)\in\LL^1(\mu_1)\te{ for }\mu_2\te{-a.e. }y\\
				y\mapsto u(x,y)\in\LL^1(\mu_2)\te{ for }\mu_1\te{-a.e. }x
			\end{aligned}
		\end{equation*}
		\item \label{x.5.b}
		\begin{equation*}
			\tag{10.10}\label{10.10}
			\begin{aligned}
				x\mapsto\int_{\Omega_2}u(x,y)\mu_2(dy)\in\LL^1(\mu_1)\\
				y\mapsto\int_{\Omega_1}u(x,y)\mu_1(dx)\in\LL^1(\mu_2)
			\end{aligned}
		\end{equation*}
		\item \label{x.5.c}\[\int_{\Omega_1\times\Omega_2}ud\mu_1\otimes\mu_2=\int_{\Omega_1}\int_{\Omega_2}u(x,y)\mu_2(dy)\mu_1(dx)=\int_{\Omega_2}\int_{\Omega_1}u(x,y)\mu_1(dx)\mu_2(dy).\label{10.11}\tag{10.11}\]
	\end{enumerate}
\end{thm}

\begin{definition}[$L^p$ spaces, $1\leq p\leq\infty$]\label{x.vi}
	For $1\leq p<\infty$, we let
	\[\LL^p(\mu)=\LL^p:=\{\func{f}{\Omega}{\K}\colon f\te{ is measurable and }|f|^p\in\LL^1\}.\tag{10.12}\label{10.12}\]
	We set
	\[\NN_p(f):=\|f\|_p:=\Big(\int_\Omega|f|^pd\mu\Big)^\frac{1}{p}.\tag{10.13}\label{10.13}\]
	For $p=\infty$, we set
	\begin{align*}
		\LL^\infty(\mu)=\LL^\infty&=\{\func{f}{\Omega}{\K}\colon f\te{ is measurable and }\mu\te{-a.e. bounded}\}\\
		&=\{\func{f}{\Omega}{\K}\colon f\te{ is measurable and }\exists C<\infty\te{ s.t. }|f(x)|\leq C\te{ for }\mu\te{-a.e. }x\in\Omega\}\\
		&=\{\func{f}{\Omega}{\K}\colon f\te{ is measurable and }\exists C<\infty\te{ s.t. }|f|\leq C\;\mu\te{-a.e.}\}
	\end{align*}
	\begin{align*}
		\NN_\infty(f):=\|f\|_\infty&:=\inf\{C\geq0\colon|f|\leq C\;\mu\te{-a.e.}\}\\
		&=:\essup_{x\in\Omega}|f(x)|\;\;\;\te{\rec{(}essential supremum\rec{)}}
	\end{align*}
\end{definition}

\begin{definition}\label{x.vii}
	If $1\leq p\leq\infty$, then the dual exponent $q$ is given by 
	\[\frac{1}{p}+\frac{1}{q}=1\tag{10.17}\label{10.17}\]
	where we set $\frac{1}{1}=0$ if $q=\infty$.
\end{definition}

\begin{thm}[Hölder]\label{x.viii}\index{measure theory!Hölder}
	Let $1\leq p\leq\infty$. If $f\in\LL^p(\mu),g\in\LL^q(\mu)$ \rec{(}or $f\in L^p,g\in L^q$\rec{)} with $p$ and $q$ dual, then
	\[fg\in\LL^1\te{ \rec{(}or }L^1\te{\rec{)} and }\|fg\|_1\leq\|f\|_p\|g\|_q.\tag{10.18}\label{10.18}\]
\end{thm}

\begin{cor}[Minkowski]\label{x.ix}\index{measure theory!Minkowski}
	Let $1\leq p\leq\infty$. $L^p(\mu)$ is a $\K$-vector space with norm $\|\cdot\|_p$.
\end{cor}

\begin{thm}[Fischer-Riesz]\label{x.x}\index{measure theory!Fischer-Riesz}
	$L^p(\mu)$ is a Banach space for $1\leq p\leq\infty$.
\end{thm}